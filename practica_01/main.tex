\documentclass[a4paper,10pt]{article}

%% Packages
\usepackage[spanish]{babel}
\usepackage[utf8]{inputenc}
\usepackage{amsmath}

%% Document
\begin{document}

\title{Teoría matemática de la relatividad general}
\author{Juan Belmonte Beitia}
\date{}

\maketitle

\begin{abstract}
El objetivo de este trabajo de investigación es introducir al lector en algunos
conceptos matemáticos usados en la investigación sobre relatividad
general, como son conceptos fundamentales de la \textbf{geometría diferencial},
el \textbf{cálculo sobre variedades} y el \textbf{álgebra tensorial}.
\end{abstract}

\section{Introducción}
En relatividad general, los objetos básicos que intervienen en una descripción
relativista de un modelo de universo son:

\begin{itemize}
\item Una \textit{variedad pseudoriemanniana}, $(\mathcal{M}, g)$ de cuatro dimensiones 
que modeliza la geometría del espacio tiempo. Aquí $g$ designa el tensor métrico y
$\mathcal{M}$ es la variedad diferenciable.

\item Los \textit{tensores} que son aplicaciones lineales definidas sobre los espacios
vectoriales tangentes y cotangentes en cada punto del espacio-tiempo. Dentro
de los tensores básicos que aparecen en la teoría de la relatividad general
están:
\begin{itemize}
\item El tensor energía-impulso, que da una descripción del tipo de materia
del que está lleno un universo, de sus propiedades básicas así como
de su distribución.
\item Los tensores de curvatura, que se pueden cálcular a partir de las
derivadas covariantes de la métrica.
\end{itemize}

\item El conjunto de observadores físicamente admisibles puede definirse como
el conjunto de posibles sístemas de coordenadas sobre la variedad espacio-tiempo.
\end{itemize}

\section{El tensor métrico}
El tensor métrico es el objeto matemético que permite calcular distancias
y otros conceptos métricos en relatividad general. Una vez especificado un 
sistema de coordenadas, el tensor métrico se representa mediante un conjunto
de funciones $g_{ij}$ , llamadas coeficicientes del tensor métrico:
\begin{equation}
\label{eq:metric}
g = \sum_{i,j=1}^n g_{ij}\ dx^i\otimes dx^i
\end{equation}

\noindent
Usando el convenio de suma de Einstein, la ecuación \eqref{eq:metric} se escribe como
$$
ds^2 = g_{ij} dx^i dx^j
$$

\noindent
El tensor métrico en el espacio de Minkowsky, expresado en coordenadas 
cartesianas, tiene como componentes $g_{ij}$ la siguiente matriz
$$
\begin{pmatrix}
-1 & 0 & 0 & 0 \\
0 & 1 & 0 & 0 \\
0 & 0 & 1 & 0 \\
0 & 0 & 0 & 1
\end{pmatrix}
$$

\noindent
Por otro lado, el hecho de que el espacio tiempo sea curvo hace que en cada
punto del espacio, los espacios vectoriales tangentes no coincidan, y por tanto al
derivar una magnitud tensorial es necesario tener en cuenta tanto la variación
de las componentes como de la base vectorial al cambiar de un punto a otro del
espacio. De esta forma, surge la definición de derivada covariante:
\begin{equation}
\label{eq:covariant}
\nabla_{\alpha} v = \nabla_{\alpha} (v^{\beta} e_{\beta}) = (\nabla_{\alpha}v^{\beta})e_{\beta} + v^{\beta}(\Gamma^{\gamma}_{\alpha \beta}e_{\gamma})
\end{equation}

\noindent
donde los coeficientes $\Gamma^{\gamma}_{\alpha \beta}$ son los llamados 
símbolos de Christoffel. Usando \eqref{eq:covariant} y el tensor de curvatura 
de Riemann, llegamos al siguiente sistema de ecuaciones:
\begin{eqnarray}
\nabla_{\alpha} \phi &=& \partial_{\alpha} \phi \label{eq:eq_01} \\
\nabla_{\alpha} v    &=& (\partial_{\alpha} V^{\gamma} + \Gamma_{\alpha \beta}^\gamma v^{\beta})e_{\gamma}  \label{eq:eq_02}
\end{eqnarray}


\subsection{La ecuación de las geodésicas}
Puede demostrarse que una curva que pasa por $x(0) = x_0$ y que en ese punto
tiene un vector tangente v es geodésica si cumple la condición:
\begin{equation}
\begin{split}
\frac{d^2x^\mu}{ds^2} + \gamma^\mu_{\sigma \nu} \frac{dx^\sigma}{ds} \frac{dx^\nu}{ds} = 0 \\
x(0) = x_0 , x_s(0) = v
\end{split}
\end{equation}

En relatividad el cálculo de las geodésicas es importante para determinar como
se está moviendo globalmente la materia en un espacio-tiempo.

\section{Tensor de Riemann o tensor de curvatura}
El tensor de curvatura se define como un tensor una vez contravariante, tres
veces covariante y viene dado, en una base natural, como

\begin{equation}
\label{eq:riemann01}
R^\rho_{\sigma \mu \nu } = \partial_\mu \Gamma^\rho_{\nu \sigma} - \partial_\nu \Gamma^\rho_{\mu \sigma} + \Gamma^\rho_{\mu \lambda} \Gamma^\lambda_{\nu \sigma} - \Gamma^\rho_{\nu \lambda} \Gamma^\lambda_{\mu \sigma}
\end{equation}

Usando \eqref{eq:riemann01} es fácil comprobar que
\begin{eqnarray*}
R_{abcd} = -R_{bacd} &=& -R_{abdc} , \\
            R_{abcd} &=& R_{cdab} , \\
            R_{a\left[bcd\right]} &=& 0
\end{eqnarray*}

De esta forma \dots

\section{Conclusiones}
En este trabajo de investigación, se ha pretendido exponer los conceptos
matemáticos más relevantes en la teoría general de la relatividad. Se han
introducido los conceptos básicos de \dots

\end{document}
